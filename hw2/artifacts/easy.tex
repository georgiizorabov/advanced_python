
\documentclass{article}
\usepackage[utf8]{inputenc}
\usepackage{array}
\usepackage[english,russian]{babel}
         
\begin{document}
\begin{center}
\hspace*{-4cm}
\begin{tabular}{ | m{10em} | m{10em} | m{10em} | m{10em} | m{10em} | }

	\hline
	Мужик & стоит & на & заправке, & заправляет \\
	\hline
	машину & и & при & этом & курит. \\
	\hline
	Тут & к & нему & подбегает & заправщик \\
	\hline
	и & давай & кричать, & дескать, & ты \\
	\hline
	что & сумасшедший!? & Кто & на & заправке \\
	\hline
	курит & да & ещё & с & открытым \\
	\hline
	баком!? & Мужик & говорит: & — & Да \\
	\hline
	ты & не & бзди. & Это & ведь \\
	\hline
	не & простые & сигареты. & Это & сигареты \\
	\hline
	термобаланс. & Заправщик & говорит, & что & ему \\
	\hline
	накласть, & как & они & называюся, & туши, \\
	\hline
	мол, & к & чёртовой & матери, & пока \\
	\hline
	не & рвануло. & Мужик & отвечает: & — \\
	\hline
	Ты & не & слышал & о & термобалансе? \\
	\hline
	Об & этом & уже & любой & ребёнок \\
	\hline
	знает. & Новейшее & открытие. & Температура & горения \\
	\hline
	табака & подстраивается & под & температуру & окружающей \\
	\hline
	среды. & Вот, & скажем, & сейчас & на \\
	\hline
	улице & +19, & значит, & и & табак \\
	\hline
	в & сигарете & загорается & при & +19. \\
	\hline
	И & количество & теплоты, & выделяемое & при \\
	\hline
	горении, & также & согласуется & с & окружающей \\
	\hline
	средой. & То & есть & сигарета & горит, \\
	\hline
	но & огонь & не & горячее & +19. \\
	\hline
	А & если, & допустим, & температура & бензина \\
	\hline
	в & баке & - & +5, & то \\
	\hline
	огонь & сигареты & не & превысит & плюс \\
	\hline
	пяти, & и & бензин & не & воспламенится, \\
	\hline
	потому & что & температура & воспламенения & бензина, \\
	\hline
	точнее & не & бензина, & а & его \\
	\hline
	паров, & как & известно, & 87 & градусов \\
	\hline
	при & попадании & источника & огня & или \\
	\hline
	электрической & искры & извне. & Не & говоря \\
	\hline
	уже & о & температуре & самовоспламенения, & которая \\
	\hline
	равняется, & если & не & ошибаюсь, & 215 \\
	\hline
	градусам. & Ну & это & на & словах \\
	\hline
	мудрёно, & а & на & деле & всё \\
	\hline
	просто. & Вот, & смотри! & И & кидает \\
	\hline
	окурок & в & бензобак. & Тут & уж \\
	\hline
	взовалось & так, & что & ни & мужика, \\
	\hline
	ни & заправщика, & ни & самой & заправки \\
	\hline
	не & осталось. \\

\hline
\end{tabular}
\end{center}
\end{document}